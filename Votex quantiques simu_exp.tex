\documentclass[a4paper]{article}

%%%%%%%%%%%%% PACKAGES UTILES %%%%%%%%%%%%%%%%%%%%%%%%%%%%%%%%%%%%%%%%%%%%

\usepackage[english]{babel}
\usepackage[utf8x]{inputenc}
\usepackage[T1]{fontenc}
\usepackage{pstricks,pstricks-add,pst-math,pst-xkey}
\usepackage{multicol}
\usepackage[a4paper,top=3cm,bottom=2cm,left=3cm,right=3cm,marginparwidth=1.75cm]{geometry}
\usepackage{amsmath}
\usepackage{graphicx}
\usepackage{graphics}
\usepackage[colorinlistoftodos]{todonotes}
\usepackage[colorlinks=true, allcolors=blue]{hyperref}
\usepackage{float}

%%%%%%%%%%%%%%%%%%%%%%%%%%%%%%%%%%%%%%%%%%%%%%%%%%%%%%%%%%%%%%%%%%%%%%%%%%

\title{De la simulation quantique à l'expérimentation : \\ les vortex quantiques}
\author{William Dubosclard, Louis Lecordier}

\begin{document}
\maketitle


\begin{abstract}
Numerical simulations usually used to study classical physics or quantum physics with many approximations don't really suite to treat these problems exactly.
Within this project we are going to take an active interest to quantum simulation with cold atoms to foresee and understand this world apart and build theories using analogy.To underline the interest of quantum simulation with cold atoms, we will study the case of the apparition of vortex in superconductor materials and compare it to simulation.
\end{abstract}

\tableofcontents

\newpage

\begin{multicols}{2}

%%%%%%%%%%%%%%%%%%%%%%%%%%%%%%%%%%%%%%%%%%%%%%%%%%%%%%%%%%%%%%%%%%%%%%%%%%%

\section{Introduction : Différents types de simulation quantique} 
\subsection*{Pourquoi la simulation quantique ?}

If you consider a numerical simulation with a computer, for a system 'as simple' as an unic atom with few electrons in an electro-magneticc field ; due to the discrete space you have to create, the computer will have to deal with $N^{3M}$ values of space with $N$ the number of points for one direction and $3M$ the number of electrons gravitating in a 3D model. These calculations are simply impossible for any computer today : For example, if one consider a $20 \times 20 \times 20$ cube of space containing a atom with only 4 electrons, it would take $20^{3 \times 4} \approx 4.10^{15}$ points that is to say $\approx 64$ petaoctets of data to stock !  It is obviously impossible to deal with with today's computers even in terms of time of calculation, it would take an infinite time to get results without any approximation like in the MCTDHF method or else... %%COURS R.TAIEB%%
R.Feynman, in a lecture about these issues, proposed that to overcome these problems, ‘quantum simulators’ that operate according to the laws of quantum mechanics should be used. If we had a system at our disposal, also composed of spin-½ particles, that we could manipulate at will, then we would be able to engineer the interaction between those particles according to the one we want to simulate, and thus predict the value of physical quantities by simply performing the appropriate measurements on our system. %%GOALS AND OPPORTUNITIES FEYNMAN%%

\subsection*{Vortex quantique et supraconductivité}
As an introduction, an abstract from a seminar given by Mithun coulb be good. Quantized vortices have long been studied in superconductors, liquid helium, BEC. observation of quantized vortices in BECs proves unambiguously the superfluidity of BECs. A rotating BEC mimics superconductor in applied magnetic field. A vortex in a BEC is a density zero region with localized phase singularity, superconductor it is the region of flux penetration. The quantization of the angular momentum demands the need of critical rotational frequency for a single vortex generation.


\section{Étude de cas vortex dans les Matériaux supraconducteurs}

Def : Type II superconductors -
Is characterized by the formation of magnetic vortices in an applied magnetic field. This occurs above a certain critical field strength $H_C1$. The vortex density increase with increasing field strength. It explains by the Meissner effect which is the expulsion of a magnetic field from a superconductor during its transition to the superconducting state.

VORTEX QUANTIZATION IN A SUPERCONDUCTOR /
The quantization in a superconductor call the Meissner effect. If the magnetic field becomes sufficiently strong, it may be energetically favorable for the superconductor to form a lattice of a quantum vortices. This lattice can be create in a Quantum Simulation with ultracold atoms. This phenomenon could be interesting to talk in our report. 
In the thesis of Ross Alexander Williams, the group observ vortex nucleation in a rotating two dimensional lattice of Bose Enstein condensates. Their experiment started with a $^{87}Rb$ condensate containing $2.10^{5}$ atoms. The atomic trap is a standard Ioffe-Pritchard configuration with a single red detuned light sheet in the radial plane increasing the axial trapping frequency. 
The experiment shows that the number of vortices observed as a function of the optical lattice rotation frequence. 


\section{Comparaison avec la simulation quantique au moyen d'un gaz ultra froid}

\textit{Observation of Vortex Lattices in Bose Einstein Condensates
}

Quantized vortices play a key role in superfluidity and superconductivity. Somes BEC groups observed the formation of highly ordered vortex lattices with a rotating Bose-condensed gas.
These observations may be a model system for study of Vortex matter.

We know that the vortex nucleation in a superconductor type II come from magnetic field which can penetrate the matter. This phenomena are direct consequences of the existence of a macroscopique wave function.

Vortices in BEC’s have been the subject of extensive theoritical study (Art : Vortices in a trapped dilute Bose-Einstein condensate, Alexander L. Fetter and Anatoly A. Svidzinsky). 
To study vortex in a Bose gas, we have to understand previously the hydrodynamic description of the condensate.

First, we represent the condensate by his wave function $\psi$ which describe a quantum-hydrodynamic function :

\[\psi(\vec{r}, t) = \psi(\vec{r}, t).\exp(i S(\vec{r}, t))\]

By using a classical hydrodynamic equation, we arrive at the analog of the Bernoulli equation for the condensate fluid with the equation :

\[\frac{1}{2}Mv^2 + V_{tr} + \frac{1}{\sqrt{n}}T\sqrt{n} + gn + M\frac{\partial\phi}{\partial t} = 0\]

We have to be careful about the gas. We are in present of a lower temperature gas. This assumption of a zero-temperature condensate implies an entropy which is negligible.
Assuming the lower entropy, we can rewrite the equation as :

\[\frac{1}{2}Mv^2 + U + \frac{e+p}{n} + M\frac{\partial \phi}{\partial t} = 0\]

Where U is the potential energy, e the energy density and e+p is the enthalpy density.
The dynamics of vortex lines at zero temperature follows from the Kelvin circulation theorem, namely that each element of the vortex core moves with the local translational velocity induced by all the sources in the fluid. It explains the vortex nucleation when the Bose-condensate rotating !

Experimentally, vortex lattices were produced by rotating the condensate around its long axis with the optical dipole force. The laser beam wavelength depend of the atom from the condensate. 

\textit{(Art : Observation of Vortex Lattices in Bose Einstein Condensates)}
We can observe in the figure \ref{fig1:v_BEC} a highly ordered triangular lattices of variable vortex density containing up to 130 vortices. Such Abrikosov lattices were first predicted for quantized magnetic flux lines in Type II superconductors (cf figure \ref{fig2:v_supra}). 
When a quantum fluid is rotated at a frequency $\Omega$, it attempts to distribute the vorticity as uniformly as possible. This is similar to a rigid body, for which the vorticity is constant.


\begin{figure}[H]
 \centering
 \includegraphics[width=\linewidth]{vortex_BEC.png}
 \caption{\label{fig1:v_BEC}Réseau de vortex dans un condensat de Bose-Einstein}
\end{figure}


\begin{figure}[H]
 \centering
 \includegraphics[width=\linewidth]{vortex_supra.png}
 \caption{\label{fig2:v_supra}Réseau de vortex dans un supraconducteur}
\end{figure}


Far from the resonance, the number of vortices produced increased with the stirring time. (Art : Vortex Formation in a Stirred Bose-Einstein Condensate K. W. Madison, F. Chevy, W. Wohlleben,* and J. Dalibard) The density of the condensate at the center of the vortex is zero, and the radius of the vortex core is of the order of the healing length $\xi = (8\pi a\rho)^{\frac{1}{2}}$.
 a is the scattering length characterizing the 2 body interaction and $\rho$ the density of the condensate. In this paper, they mentioned the lifetime of a vortex.


\begin{figure}[H]
 \centering
 \includegraphics[width=\linewidth]{graph_BEC.png}
 \caption{\label{fig3:graph_BEC}Probabilité de survie des vortex en fontion du temps}
\end{figure}

Fraction of images \ref{fig3:graph_BEC} showing a vortex as a function of the time spent by the gas in the axisymmetric trap after the end of the stirring phase for two condensate conditions.

 The vortex lattice had lifetimes of several seconds. Moreover, the lifetime of vortices markedly decreased with the number of condensed atoms. Theoretical calculation isn’t approach in this report but it presents in the article (Art : Dissipative dynamics of vortex arrays in trapped Bose-Condensed gases : neutron stars physics on $\mu K$ scale P.O. Fedichev and A.E. Muryshev)

Properties of vortex lattices are of broad interest in superfluids, superconductors, and even astrophysics. Fluctuations in the rotation rate of pulsars are attributed to the dynamics of the vortex lattice in a superfluid neutron liquid. Vortex formation and self-assembly into a regular lattice is a robust feature of rotating BECs. Gaseous condensates may serve as a model system to study the dynamics of vortex matter, in analogy to work in type II superconductors.



%\begin{figure}[H]
% \centering
% \includegraphics[width=\linewidth]{vortex_supra_reseau.jpeg}
% \caption{\label{fig:vortex}my caption of the figure}
%\end{figure}

\end{multicols}
\end{document}
